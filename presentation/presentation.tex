% Copyright 2004 by Till Tantau <tantau@users.sourceforge.net>.
%
% In principle, this file can be redistributed and/or modified under
% the terms of the GNU Public License, version 2.
%
% However, this file is supposed to be a template to be modified
% for your own needs. For this reason, if you use this file as a
% template and not specifically distribute it as part of a another
% package/program, I grant the extra permission to freely copy and
% modify this file as you see fit and even to delete this copyright
% notice. 

\documentclass{beamer}

% There are many different themes available for Beamer. A comprehensive
% list with examples is given here:
% http://deic.uab.es/~iblanes/beamer_gallery/index_by_theme.html
% You can uncomment the themes below if you would like to use a different
% one:
%\usetheme{AnnArbor}
%\usetheme{Antibes}
%\usetheme{Bergen}
%\usetheme{Berkeley}
%\usetheme{Berlin}
%\usetheme{Boadilla}
%\usetheme{boxes}
%\usetheme{CambridgeUS}
%\usetheme{Copenhagen}
%\usetheme{Darmstadt}
\usetheme{default}
%\usetheme{Frankfurt}
%\usetheme{Goettingen}
%\usetheme{Hannover}
%\usetheme{Ilmenau}
%\usetheme{JuanLesPins}
%\usetheme{Luebeck}
%\usetheme{Madrid}
%\usetheme{Malmoe}
%\usetheme{Marburg}
%\usetheme{Montpellier}
%\usetheme{PaloAlto}
%\usetheme{Pittsburgh}
%\usetheme{Rochester}
%\usetheme{Singapore}
%\usetheme{Szeged}
%\usetheme{Warsaw}

\usepackage{graphicx}
\graphicspath{{../figures/}}

\title{Presentation Title}

% A subtitle is optional and this may be deleted
\subtitle{Optional Subtitle}

\author{K.~Cong}
% - Give the names in the same order as the appear in the paper.
% - Use the \inst{?} command only if the authors have different
%   affiliation.

\institute[Delft University of Technology] % (optional, but mostly needed)
{
  Faculty of Electrical Engineering, Mathematics and Computer Science\\
  Delft University of Technology}
% - Use the \inst command only if there are several affiliations.
% - Keep it simple, no one is interested in your street address.

% \date{Conference Name, 2013}
\date{2017}
% - Either use conference name or its abbreviation.
% - Not really informative to the audience, more for people (including
%   yourself) who are reading the slides online

% \subject{Theoretical Computer Science}
% This is only inserted into the PDF information catalog. Can be left
% out. 

% If you have a file called "university-logo-filename.xxx", where xxx
% is a graphic format that can be processed by latex or pdflatex,
% resp., then you can add a logo as follows:

% \pgfdeclareimage[height=0.5cm]{university-logo}{university-logo-filename}
% \logo{\pgfuseimage{university-logo}}

% Delete this, if you do not want the table of contents to pop up at
% the beginning of each subsection:
\AtBeginSubsection[]
{
  \begin{frame}<beamer>{Outline}
    \tableofcontents[currentsection,currentsubsection]
  \end{frame}
}

% Let's get started
\begin{document}

\begin{frame}
  \titlepage
\end{frame}

\begin{frame}{Outline}
  \tableofcontents
  % You might wish to add the option [pausesections]
\end{frame}

% Section and subsections will appear in the presentation overview
% and table of contents.
\section{Background}

\subsection{TrustChain}

\begin{frame}{TrustChain}{}

  \begin{figure}[h]
  \includegraphics[width=0.7\textwidth]{trustchain-good}
  \centering
  \caption{Every node has a chain. \emph{TX block} is a six-tuple: $t_{i,j} =
    (\texttt{h}(b_{i,j-1}), h_s, h_r, s_s, s_r, m)$, one transaction results in
    two TX blocks---\emph{a pair}.}
  \end{figure}

\end{frame}

\begin{frame}{TrustChain}{}

  \begin{figure}[h]
  \includegraphics[width=0.7\textwidth]{trustchain-bad}
  \centering
  \caption{Fork is two correctly signed TX blocks that belong to the same sender
    and has the same hash pointer but involve different receivers. Only one TX
    block may be in consensus.}
  \end{figure}

\end{frame}

\begin{frame}{TrustChain}
  \begin{itemize}
    \item Everyone has their own chain
    \item Transactions are on arbitrary data $m$
    % \item Transactions make the chains intertwined
    \item Transactions are irrefutable due to hash pointers
    \item No consensus (my thesis)
  \end{itemize}
\end{frame}

\section{My Thesis}
\subsection{TrustChain with Checkpoints}
\begin{frame}{TrustChain with Checkpoints}

  \begin{figure}[h]
  \includegraphics[width=0.7\textwidth]{trustchain-bad-cp}
  \centering
  \caption{CP block is a six-tuple: $c_{i,j} = (\texttt{h}(b_{i,j-1}),
    \texttt{h}(\mathcal{C}_r), h, r, p, s)$, $\mathcal{C}_r$ is the consensus
    result at round $r$, $p =$ promoter indicator, $s =$ signature.}
  \end{figure}

\end{frame}

\subsection{Protocol Overview}
\begin{frame}{Protocol Overview}
  \begin{enumerate}
    \item $n$ lucky nodes are selected at random to act as promoters.
    \item Promoters run a BFT (Byzantine Fault Tolerant) consensus algorithm to
      agree on a set of CP blocks.
    \item Disseminate the consensus result (the CP blocks).
    \item Repeat for next round.
    \item Any interested node can validate that their transaction.
  \end{enumerate}
\end{frame}

\subsection{Promoter Registration}
\begin{frame}{Promoter Registration}

  \begin{figure}[h]
  \includegraphics[width=0.7\textwidth]{trustchain-1}
  \centering
  \caption{We start in the state where $\mathcal{C}_{r - 1}$ has just been agreed but
    not yet propagated.}
  \end{figure}

\end{frame}

\begin{frame}{Promoter Registration}

  \begin{figure}[h]
  \includegraphics[width=0.7\textwidth]{trustchain-2}
  \centering
  \caption{Nodes receive consensus result and set promoters indicator $p$, then
    send the new CP blocks to promoters of round $r$.}
  \end{figure}

\end{frame}

\begin{frame}{Promoter Registration}

  \begin{figure}[h]
  \includegraphics[width=0.7\textwidth]{trustchain-3}
  \centering
  \caption{Transactions carry on as usual in round $r$. Note that our CP blocks
    (round $r - 1$) has not reached consensus yet.}
  \end{figure}

\end{frame}

\begin{frame}{Promoter Registration}

  \begin{figure}[h]
  \includegraphics[width=0.7\textwidth]{trustchain-4}
  \centering
  \caption{CP blocks at round $r-1$ should be in $\mathcal{C}_r$. If we're
    lucky: $\texttt{h}(k_i || c_{i,j}) < T$, then we're responsible for
    consensus of round $r + 1$.}
  \end{figure}

\end{frame}

\subsection{Consensus}
\begin{frame}{Consensus}
  \begin{enumerate}
    \item Nodes send CP blocks to the promoters.
    \item The promoters' identities are encoded in the consensus result.
    \item Promoters run the some asynchronous BFT consensus algorithm to agree on
      a set of CP blocks---$\mathcal{C}_r$.
    \item $n \ge 3t + 1$ is the optimal for BFT consensus.
    \item $\mathcal{C}_r$ and the signatures are disseminated.
    \item Nodes create new CP blocks when they receive $t + 1$ good signatures and
      $\mathcal{C}_r$.
  \end{enumerate}

  \begin{theorem}
    Assuming the promoters satisfy $n \ge 3t + 1$. The promoter registration and
    the consensus protocol satisfies agreement, total order and liveness.
  \end{theorem}
\end{frame}

\subsection{Validation}

\begin{frame}{Validation}
Assume node $u$ is aware of all the past consensus results $\mathcal{C}_r$.
Suppose $u$ wish to validate $t_{i,j}$. It performs the following.

\begin{enumerate}
\item Determine the pair $t_{i', j'}$.
\item Find the agreed enclosure for $t_{i,j}$ and $t_{i', j'}$ from
  $\mathcal{C}_r$, otherwise return ``unknown''.
%\item Find the smallest $c_{i', b}$ such that $b > j'$ is in consensus.
\item Query $i$ and $i'$ for the agreed pieces and ensure hash pointers are
  correct. Otherwise return ``unknown''.
\item Check that $t_{i,j}$ and $t_{i', j'}$ are in the agreed pieces and are
  created correctly using \texttt{newtx}. Otherwise return ``invalid''.
\item Check the checkpoints $c_{i, k}$ and $c_{i', k'}$ that immediately follow
  $t_{i,j}$ and $t_{i', j'}$ are in the agreed pieces and are created in the
  same round, i.e. $\texttt{round}(c_{i, k}) = \texttt{round}(c_{i', k'})$.
  Otherwise return ``invalid''.
\item Return ``valid''.
\end{enumerate}
\end{frame}

\begin{frame}{Validation}
  In essence, given a TX, ask the receiver to proof that it has a set of
  transactions that produces some CP block, the CP block should be in the
  consensus result and the pair of the TX should be in that set.

  \begin{theorem}
    If at least one party of every transaction is honest, then forking and other
    forms of tampering is guaranteed to be detected if the malicious party is
    alive.
  \end{theorem}

  \begin{block}{Question}
    Given a starting digest and an ending digest, is it possible for the prover
    to proof to the verifier that it has a set of blocks linked by hash pointers
    that ``fits'' inside those digests in \emph{zero knowledge}.
  \end{block}
\end{frame}

\subsection{Implementation and Experimentation Results}
\begin{frame}{Implementation (4 weeks ago)}
  \begin{itemize}
    \item Currently on going---using Python and Twisted
    \item Completed BFT consensus algorithm
    \item Completed local TrustChain with Checkpoints
    \item Next step is networked TrustChain and validation protocol
  \end{itemize}
\end{frame}

\begin{frame}{Implementation}
  \begin{itemize}
    \item Completed BFT consensus algorithm, with erasure coding
    \item Completed TrustChain with Checkpoints, with experiments on DAS5
    \item Validation function is working
    \item Next step is to build validation into a network protocol and do experiments
  \end{itemize}
\end{frame}

\begin{frame}{Experiment with 500 nodes---no gossip}
  \begin{figure}[h]
  \includegraphics[width=0.95\textwidth]{consensus-500-no-gossip}
  \centering
  \end{figure}
\end{frame}

\begin{frame}{Experiment with 500 nodes---with gossip}
  \begin{figure}[h]
  \includegraphics[width=0.95\textwidth]{consensus-500}
  \centering
  \end{figure}
\end{frame}

\begin{frame}{Experiment with 500 nodes---with gossip}
  \begin{block}{Conclusion}
    \begin{itemize}
      \item The number of facilitators affects the consensus latency ($O(N\log{N})$?).
      \item The transaction rate has little or not effect on the average consensus latency.
    \end{itemize}
  \end{block}
\end{frame}

\begin{frame}{Planned experiments}
  \begin{itemize}
    \item Consensus latency (y-axis) vs population size (x-axis)---to
      investigate how population size impacts consensus latency.
    \item Global transaction rate vs population size---to investigate
      the scalability property.
    \item Global validation rate vs population size---to also investigate the
      scalability property.
  \end{itemize}
\end{frame}
\end{document}


