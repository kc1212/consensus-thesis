\section{System Architecture}
\label{sec:system}

To describe \textsc{Checo}, we first give an informal overview and then move on to the formal description.

Early blockchain systems that use a global ledger are difficult to scale because every node must reach consensus on all the transactions that ever existed.
Instead, we introduce an alternative architecture where every node has their own genesis block and hash chain.
The nodes only store transactions (TX) that they are involved in on their hash chains.
Transactions are stored in TX blocks, and every block only contains one transaction.
A transaction between two nodes should, therefore, result in two TX blocks on their respective hash chains.
We introduce a special block called checkpoint (CP) block,
which represents the state of a hash chain in the form of a hash pointer.
Then, a collection of CP blocks from all nodes would represent the state of the whole system.
A visualisation can be seen in~\Cref{fig:trustchain-good-cp}.
\begin{figure}[h]
\centering
\includegraphics[width=\linewidth]{trustchain-good-cp}
\caption{Visualisation of the data structure used in \textsc{Checo}.
$t_{u, i}$ represents a TX block on $u$'s chain with a sequence number $i$.
$c_{v, j}$ represents a CP block on $v$'s chain with a sequence number $j$.
The blocks at the ends of the dotted lines are pairs of each other.
Blocks of sequence number 0 (e.g., $t_{c, 0}$) are genesis blocks.}
\label{fig:trustchain-good-cp}
\end{figure}

\textsc{Checo} consists of three protocols---consensus protocol, transaction protocol and validation protocol---all interacting with the distributed hash chain data structure described above.
The primary protocol is the consensus protocol,
which can be seen as a technique of running infinitely many times of an existing Byzantine consensus algorithm (in this work we use asynchronous common subset from HoneyBadgerBFT~\cite{miller2016honey}),
starting a new execution immediately after the previous one is completed.
Nodes create new CP blocks at the end of every execution.
This approach is necessary because blockchain systems always need to reach consensus on new values proposed by the nodes in the system,
or CP blocks in our case.

The communication complexity of Byzantine consensus algorithms typically grows polynomially w.r.t the number of nodes,
which prohibits us from running it on a large network.
Thus, at the beginning of every Byzantine consensus algorithm execution,
we randomly elect a set of nodes---called facilitators---to collect CP blocks from every other node and use those blocks as the input to the Byzantine consensus algorithm.
After the algorithm completes,
the facilitators output a set of CP blocks which we call the consensus result,
which is then propagated to the network.
Using the result, nodes are allowed to create new CP blocks,
and then the next algorithm execution begins.

The transaction protocol is a simple request and response protocol.
The nodes exchange one round of messages and create new TX blocks on their respective chains.
Thus, as we mentioned before, one transaction should result in two TX blocks.

The consensus and transaction protocol by themselves do not provide a mechanism to detect malicious behaviour such as tampering.
Thus, we need a validation protocol to counteract such behaviour.
When a node wishes to validate one of its transactions, it asks the counterparty for the \emph{agreed fragment} of the transaction.
Which is a section of the counterparty's chain beginning and ending with CP blocks but contains the TX block belonging to that transaction,
where the CP blocks must be in consensus.
Upon the counterparty's response,
the node checks whether the CP blocks are, in fact, in some consensus result and among other conditions.
The transaction is valid if these conditions are satisfied.
% Intuitively, this works because it is hard (because a cryptographically secure hash function is collision resistant)
% to create a different chain that begins and ends with the same two CP blocks.
Since the transaction and validation protocols only make point-to-point communication,
we achieve horizontal scalability.

For brevity, we left out issues such as selecting facilitators in a fair way and the exact validity condition.
The following sections give the formal description and will answer these issues.

\subsection{\textsc{Checo} data structure}
Each node $u$ has a public and private key pair---$pk_u$ and $sk_u$, and a hash chain $B_u$.
The chain consist of blocks $B_u = \{ b_{u, i} : i \in \{ 0, \dots, h - 1 \} \}$,
where $b_{u, i}$ is the $i$th block of $u$,
and $h$ is the height of the block (i.e. $h = |B_u|$).
We use $b_{u, h-1}$ to denote the latest block.
There are two types of blocks, TX blocks and CP blocks.
If $T_u$ is the set of all TX blocks in $B_u$ and $C_u$ is the set of all CP blocks is $B_u$,
then $T_u \cup C_u = B_u$ and $T_u \cap C_u = \varnothing$.
The notation $b_{u, i}$ is generic over the block type.

\begin{definition}
\textbf{Transaction block}

The TX block is a six-tuple, i.e
$$t_{u, i} = \langle \textsf{H}(b_{u, i - 1}), seq_u, txid, pk_v, m, sig_u \rangle.$$
We describe each item in turn.
\begin{enumerate}
\item $\textsf{H}(b_{u, i - 1})$ is the hash pointer to the previous block.
\item $seq_u$ is the sequence number which should equal $i$.
\item $txid$ is the transaction identifier, it should be generated using a cryptographically secure pseudo-random number generator by the initiator of the transaction.
\item $pk_v$ is the public key of the counterparty $v$.
\item $m$ is the transaction message,
    which can be seen as an arbitrary string.
\item $sig_u$ is the signature created using $sk_u$ on the concatenation of the binary representation of the five items above.
\end{enumerate}
% The fact that we have no constraint on the content of $m$ is in alignment with our problem description---application neutrality.

TX blocks come in pairs.
In particular, for every block 
$$t_{u, i} = \langle \textsf{H}(b_{u, i - 1}), seq_u, txid, pk_v, m, sig_u \rangle$$
there exists one and only one \emph{pair} 
$$t_{v, j} = \langle \textsf{H}(b_{v, j - 1}), seq_v, txid, pk_u, m, sig_v \rangle,$$
if the nodes follow the transaction protocol (described in~\Cref{sec:tx-protocol}).
Note that the $txid$ and $m$ are the same, and the public keys refer to each other.
Thus, given a TX block, these properties allow us to identify its pair.
\end{definition}

% TODO Node $v$ may cheat and create more than one pair for $t_{u, i}. we discuss later

\begin{definition}
\textbf{Checkpoint block and genesis block}

The CP block is a five-tuple, i.e. 
$$c_{u, i} = \langle \textsf{H}(b_{u, i-1}), seq_u, \textsf{H}(\C_r), r, sig_u \rangle,$$
where $\C_r$ is the consensus result (which we describe next in \Cref{sec:consensus-result}) in round $r$,
the other items are the same as the TX block definition.

The genesis block in the chain must be a CP block in the form of
$$c_{u, 0} = \langle \textsf{H}(\epsilon), 0,  \textsf{H}(\epsilon), 0, sig_u \rangle,$$
where $\epsilon$ is the empty string.
The genesis block is unique because every node has a unique public and private key pair.
\end{definition}

\begin{definition}
\label{sec:consensus-result}
\textbf{Consensus result}

Our consensus protocol runs in rounds,
where the first round is defined to be 1 and it is incremented after every execution of the consensus protocol.
The consensus result, output of the consensus protocol, is a tuple, i.e. 
$$\C_r = \langle r, C \rangle,$$
where $C$ is a set of CP blocks agreed by the facilitators of round $r$.
\end{definition}

Next we define a property which results from the interleaving nature of CP and TX blocks.
It is used in our validation protocol (discussed in~\Cref{sec:vd-protocol}).

\begin{definition}
\textbf{Enclosure and agreed enclosure}

If there exists a tuple $\langle c_{u,a}, c_{u, b} \rangle$ for a TX block $t_{u, i}$, where
\begin{itemize}
\item $c_{u, a}$ is the closest CP block to $t_{u, i}$ with a lower sequence number and
\item $c_{u, b}$ is the closest CP block to $t_{u, i}$ with a higher sequence number,
\end{itemize}
then $\langle c_{u,a}, c_{u, b} \rangle$ is the enclosure of $t_{u, i}$.
Some TX blocks may not have any enclosure, then their enclosure is $\bot$.
Agreed enclosure is the same as enclosure with an extra constraint where the CP blocks must be in some consensus result $\C_r$.
\end{definition}

\begin{definition}
\textbf{Fragment and agreed fragment}

If the enclosure of some TX block $t_{u, i}$ is $\langle c_{u,a}, c_{u, b} \rangle$,
then its fragment $F_{u, i}$ is defined as $\{ b_{u, i} : a \le i \le b \}$.
Similarly, agreed fragment has the same definition as fragment but using agreed enclosure.
For convenience, the function $\textsf{agreed\_fragment}(t_{u, i})$ outputs the agreed fragment of $t_{u, i}$ if it exists, otherwise $\bot$.
\end{definition}

\subsection{Consensus Protocol}
Our scalable consensus protocol $\Pi_c$ uses an unmodified asynchronous common subset (ACS) algorithm as the key building block.
The objectives of the protocol are to
allow honest nodes always make progress (in the form of creating new CP blocks),
compute correct consensus result in every round
and have an unbiased election of facilitators.
We formally define the desired properties below.
\begin{definition}
\label{def:consensus}
\textbf{\textsc{Checo} consensus protocol}

A \textsc{Checo} consensus protocol is correct if the following holds for every round $r$.
\begin{enumerate}
    \item \emph{Agreement}:
        If one correct node outputs a set of facilitators $\F_r$,
        then every node outputs $\F_r$
    \item \emph{Validity}:
        If any correct node outputs $\F_r$, then 
            \begin{enumerate}
                \item $|\C_r| \ge N - t$\footnote{
                $\C_r$ is a tuple but we abuse the notation here by writing $|\C_r|$ to mean the number of CP blocks in the second element of $\C_r$.}, and
                \item $|\F_r| = n$.
            \end{enumerate}
    \item \emph{Fairness}:
        Every node with a CP block in $\C_r$ should have an equal probability of becoming a member of $\F_r$.
    \item \emph{Termination}:
        Every correct node eventually outputs some $\F_r$.
\end{enumerate}
\end{definition}

\subsubsection{Bootstrap Phase}
\label{sec:bootstrap}
To bootstrap, imagine that there is some bootstrap oracle that initiates the correct program on every node,
meaning that it satisfied the properties in \Cref{def:consensus}.
In practice, the bootstrap oracle is most likely the software developer that sets up the system and assigns the facilitators of round 1 in a fair way.
This concludes the bootstrap phase.
For any future rounds, the consensus phase is used.

\subsubsection{Consensus Phase}
\label{sec:consensus-phase}

For any node $u$,
the consensus phase begins when $\F_r$ is available and the latest block is $c_{u, h-1}$.
Note that $\F_r$ indicates the facilitators that were elected using results of round $r$ and are responsible for driving the ACS algorithm in round $r + 1$.
The goal is to reach agreement on a set of new facilitators $\F_{r + 1}$ that satisfies the four properties in \Cref{def:consensus}.

There are two scenarios in the consensus phase.
First, if $u$ is not the facilitator, it sends $\langle \texttt{cp\_msg}, c_{u, h-1} \rangle$ to all the facilitators.
Second, if $u$ is a facilitator, it waits until $N - t$ messages of type \texttt{cp\_msg} are received.
% Second if the node is a facilitator, it waits for some duration $D$ and collect messages of type \texttt{cp\_msg}, where $D \gg \Delta$.
Invalid messages are removed.
Those are blocks with invalid signatures and blocks signed by the same key.
With the sufficient number of \texttt{cp\_msg} messages,
it begins the ACS algorithm and some $\C'_{r + 1}$ should be agreed upon by the end of it.
Duplicates and blocks with invalid signatures are again removed from $\C'_{r+1}$ and we call the final result $\C_{r+1}$.
We have to remove invalid blocks a second time because the adversary may send different CP blocks to different facilitators,
which results in invalid blocks in the ACS output, but not in any of the inputs.

The core of the consensus phase is the ACS algorithm,
which is derived from HoneyBadgerBFT~\cite{miller2016honey}.
We do not use the full HoneyBadgerBFT due to the following.
First, the transactions in HoneyBadgerBFT are first queued in a buffer and the main consensus algorithm starts only when the buffer reaches an optimal size.
We do not have an infinite stream of CP blocks, thus buffering is unsuitable.
Second, HoneyBadgerBFT uses threshold encryption to hide the content of the transactions.
But we do not reach consensus on transactions, only CP blocks, so hiding CP blocks is meaningless for us.

Continuing, when $\F_{r}$ finish the ACS execution and reach agreement on $\C_{r+1}$,
they immediately broadcast two messages to all the nodes---first the consensus message $\langle \texttt{cons\_msg}, \C_{r+1} \rangle$,
and second the signature message $\langle \texttt{cons\_sig}, r, sig \rangle$.
The reason for sending \texttt{cons\_sig} is the following.
The channels are not authenticated, 
and there are no signatures in $\C_{r + 1}$.
If a non-facilitator sees some $\C_{r + 1}$, it cannot immediately trust it because it may have been forged.
Thus, to guarantee authenticity, every facilitator sends an additional message that is the signature of $\C_{r + 1}$.

Upon receiving $\C_{r + 1}$ and at least $n - t$ valid signatures, $u$ performs two tasks.
First, it creates a new CP block using $\textsf{new\_cp}(\C_{r + 1})$ (\Cref{alg:new-cp}).
Second, it computes the new facilitators using $\textsf{get\_facilitator}(\C_{r + 1}, n)$ (\Cref{alg:facilitator}),
and updates its facilitator set to the result.
This concludes the consensus phase and brings us back to the state at the beginning of the consensus phase,
so a new round can be started.

Our protocol has some similarities with synchronizers~\cite[Chapter 10]{podc} because it is effectively a technique to introduce synchrony in an asynchronous environment.
If we consider the facilitators as a collective authority, then it can be seen as a synchronizer that sends pulse messages (in the form of \texttt{cons\_msg} and \texttt{cons\_sig}) to indicate the start of a new clock pulse.
Every node then sends a completion messages (in the form of \texttt{cp\_msg}) to the new collective authority to indicate that they are ready for the next pulse.

\begin{algorithm}
\caption{Function $\textsf{new\_cp}(\C_r)$ runs in the context of the caller $u$.
It creates a new CP block and appends it to $u$'s chain.}
\label{alg:new-cp}
\begin{algorithmic}
\State $\langle r, \_ \rangle \gets \C_r$
\State $h \gets |B_u|$
\State $c_{u, h} \gets \langle \textsf{H}(b_{u, h-1}), h, \textsf{H}(\C_r), r, sig_u \rangle$
\State $B_u \gets B_u \cup c_{u, h}$
\end{algorithmic}
\end{algorithm}

\begin{algorithm}
\caption{Function $\textsf{get\_facilitator}(\C_r, n)$ takes the consensus result $\C_r$ and an integer $n$,
then sorts the CP blocks $C$ by the luck value (the $\lambda$-expression), and outputs the smallest $n$ elements.}
\label{alg:facilitator}
\begin{algorithmic}
\State $\langle r, C \rangle \gets \C_r$
\State \Return $\textsf{take} (n, \textsf{sort\_by} (\lambda x.\textsf{H}(\C_r || pk \text{ of } x), C))$
\end{algorithmic}
\end{algorithm}

\subsection{Transaction Protocol}
\label{sec:tx-protocol}

The TX protocol $\Pi_t$, shown in \Cref{alg:tx-proto}, is run by all nodes.
Nodes that wish to initiate a transaction calls $\textsf{new\_tx}(pk_v, m, txid)$ (\Cref{alg:new-tx}) with the intended counterparty $v$ identified by $pk_v$ and message $m$.
$txid$ should be a uniformly distributed random value, i.e. $txid \in_R \{0, 1\}^{256}$.
Then the initiator sends $\langle \texttt{tx\_req}, t_{u, h}\rangle$ to $v$.

\begin{algorithm}
    \caption{Function $\textsf{new\_tx}(pk_v, m, txid)$ generates a new TX block and appends it to the caller $u$'s chain.
    It is executed in the private context of $u$, i.e. it has access to the $sk_u$ and $B_u$.}
    \label{alg:new-tx}

    \begin{algorithmic}
    \State $h \gets |B_u|$
    \State $t_{u, h} \gets \langle \textsf{H}(b_{u, h - 1}), h, txid, pk_v, m, sig_u \rangle$
    \State $B_u \gets B_u \cup \{ t_{u, h} \}$
    \end{algorithmic}
\end{algorithm}

\begin{algorithm}
    \caption{$\Pi_t$ runs in the context of node $u$.}
    \label{alg:tx-proto}

    \begin{algorithmic}
        \Upon $\langle \texttt{tx\_req}, t_{v, j} \rangle$ from $v$
        \State $\langle \_, \_, txid, pk_v, m, \_ \rangle \gets t_{v, j}$
        \State $\textsf{new\_tx}(pk_u, m, txid)$
        \State store $t_{v, j}$ as the pair of $t_{u, h}$
        \State send $\langle \texttt{tx\_resp}, t_{u, h} \rangle$ to $v$

        \Upon $\langle \texttt{tx\_resp}, t_{v, j} \rangle$ from $v$
        \State $\langle \_, \_, txid, pk_v, m, \_ \rangle \gets t_{v, j}$
        \State store $t_{v, j}$ as the pair of the TX with identifier $txid$
    \end{algorithmic}
\end{algorithm}

A key feature of $\Pi_t$ is that it is non-blocking.
At no time in \Cref{alg:new-tx} or \Cref{alg:tx-proto} do we need to hold the chain state and wait for some message to be delivered before committing a new block to the chain.
This allows for a high level of concurrency where we can call $\textsf{new\_tx}(\cdot)$ and send \texttt{tx\_req} messages multiple times without waiting for the corresponding \texttt{tx\_resp} messages.

\subsection{Validation Protocol}
\label{sec:vd-protocol}

Up to this point, we do not provide a mechanism to detect tampering.
The validation protocol $\Pi_v$ aims to solve this issue.
The protocol is also a request-response protocol.
But before explaining the protocol itself, we first define what it means for a transaction to be valid.

\subsubsection{Validity Definition}
A transaction can be in one of three states in terms of validity---\emph{valid}, \emph{invalid} and \emph{unknown}.
Given a fragment $F_{v, j}$,
the validity of the TX block $t_{u, i}$ with its corresponding fragment $F_{u, i}$ is captured by the function $\textsf{get\_validity}(t_{u, i}, F_{u, i}, F_{v, j})$ (\Cref{alg:get-validity}).
Note that $t_{u, i}$ and $F_{u, i}$ are assumed to be valid,
otherwise the node calling the function would have no point of reference.
This is not difficult to achieve because typically the caller is $u$,
so it knows its own TX block and the corresponding agreed fragment.
If the caller is not $u$, it can always query for the agreed fragment that contains the transaction of interest from $u$.

% \Cref{alg:get-validity} works as follows.
% Before~\Cref{line:valid-fragment},
% we essentially check whether the fragment is the one that the verifier needs.
% If it is not, then the verifier cannot make any decision and return \emph{unknown}.
% This is likely to be the case for new transactions because the result of $\textsf{agreed\_fragment}(\cdot)$ would be $\bot$.
% The next two conditions check for tampering or missing blocks, if any of these misconducts are detected, then the TX is invalid.

We stress that the \emph{unknown} state means that the verifier does not have enough information to make progress in $\Pi_v$.
If enough information is available at a later time, then the verifier will output either \emph{valid} or \emph{invalid}.

Note that the validity is on a transaction, i.e. two TX blocks that form a pair.
It is defined this way because the malicious sender may create new TX blocks in their own chain but never send \texttt{tx\_req} messages.
In that case, it may seem that the counterparty, who is honest, purposefully omitted TX blocks.
But in reality, it was the malicious sender who did not follow the protocol.
Thus, in such cases, the whole transaction identified by its $txid$ is marked as invalid.

\begin{algorithm}
\caption{Function $\textsf{get\_validity}(t_{u, i}, F_{u, i}, F_{v, j})$ validates the transaction represented by $t_{u, i}$.
We assume $F_{u, i}$ is always correct and contains $t_{u, i}$.
$F_{v, j}$ is the corresponding fragment received from $v$.}
\label{alg:get-validity}

\begin{algorithmic}

    \If{$F_{v, j}$ is not a fragment created in the same round as $F_{u, i}$}
        \State \Return \emph{unknown}
    \EndIf

    \label{line:valid-fragment}
    \State $\langle \_, \_, txid, pk_v, m, \_ \rangle \gets t_{u, i}$
    \If{number of blocks of $txid$ in $F_{v, j} \ne 1$}
        \State \Return \emph{invalid}
    \EndIf

    \State $t_{v, j} \gets $ the TX block with $txid$ in $F_{v, j}$
    \State $\langle \_, \_, txid', pk_u, m', \_ \rangle \gets t_{v, j}$ 

    % message tampering
    \If{$m \ne m' \vee txid \ne txid'$} 
        \State \Return \emph{invalid}
    \EndIf

    % signature tampering
    \If{$t_{u, i} \text{ is not signed by } pk_u \vee t_{v, j} \text{ is not signed by } pk_v $}
        \State \Return \emph{invalid}
    \EndIf
    \State \Return \emph{valid}
\end{algorithmic}
\end{algorithm}

\subsubsection{Validation Protocol}
Our validation protocol $\Pi_v$,
shown in~\Cref{alg:vd-proto},
is designed to classify transactions according to the aforementioned validity definition.
If $u$ wishes to validate some TX with ID $txid$ and counterparty $v$, it sends $\langle \texttt{vd\_req}, txid \rangle$ to $v$.
The desired properties are as follows.
\begin{definition}
\label{def:validation}
\textbf{\textsc{Checo} validation protocol}

A \textsc{Checo} validation protocol is correct if the following properties hold.
\begin{enumerate}
    \item \emph{Agreement}:
        If any correct node decides on the validity of a transaction, except when it is \emph{unknown},
        then all other correct nodes are able to reach the same conclusion or decide \emph{unknown}.
    \item \emph{Validity}:
        The validation protocol outputs the correct result
        according to the validity definition above.
    % \item \emph{Unforgeability}:
    %     If some transaction is valid, it cannot be forged into an invalid transaction.
    %     If some transaction is invalid, it cannot be forged into a valid transaction.
    \item \emph{Liveness}:
        Any valid (invalid) transaction is marked as valid (invalid) eventually.
\end{enumerate}
\end{definition}

\begin{algorithm}
\caption{$\Pi_v$ which runs in the context of $u$}
\label{alg:vd-proto}

\begin{algorithmic}
    \Upon $\langle \texttt{vd\_req}, txid \rangle$ from $v$
        \State $t_{u, i} \gets \text{the transaction identified by } txid$
        \State $F_{u, i} \gets \textsf{agreed\_fragment}(t_{u, i})$
        \State send $\langle \texttt{vd\_resp}, txid, F_{u, i} \rangle$ to $v$

    \Upon $\langle \texttt{vd\_resp}, txid, F_{v, j} \rangle$ from $v$
        \State $t_{u, i} \gets \text{the transaction identified by } txid$
        \If{$F_{u, i}$ and $F_{v, j}$ are available and $F_{u, i}$ is the agreed fragment of $t_{u, i}$}
            \State set the validity of $t_{u, i}$ to $\textsf{get\_validity}(t_{u, i}, F_{u, j}, F_{v, j})$
        \EndIf
\end{algorithmic}
\end{algorithm}

We make two remarks.
First, just like $\Pi_t$, we do not block any part of the protocol.
Second, suppose some $F_{v, j}$ validates $t_{u, i}$, then that does not imply that $t_{u, i}$ only has one pair $t_{v, j}$.
Our validity requirement only requires that there is only one $t_{v, j}$ in the correct consensus round.
The counterparty may create any number of fake pairs in later consensus rounds.
But these fake pairs only pollutes the chain of $v$ and can never be validated.

\section{Design Variations and Tradeoffs}
\label{sec:tradeoffs}
In this section, 
we explore a few design variations, some of them require a relaxed version of our original model.
They enable better performance and allow us to apply our design in the fully permissionless setting.

\subsection{Open membership using timing assumption}
\label{sec:permissionless}
At the start of our consensus phase (\Cref{sec:consensus-phase}), facilitators must wait for $N-f$ \texttt{cp\_msg} messages.
The use of $N$ makes our system unsuitable for the open membership setting,
where nodes may join and leave at will (churn).
We over come this problem by introducing a timing assumption.
Concretely, instead of waiting for $N-f$ messages, we wait for some time $D$,
such that $D$ is sufficiently long for honest nodes to send their CP blocks to the facilitators.
Consequently, this removes the need for a PKI because the collected CP blocks may be from nodes that nobody has seen in the past.

The new protocol handles churn as follows.
Suppose a new node wish to join the network and the facilitators are known (this can be done with a public registry).
It simply sends its latest CP block to the facilitators.
Then, in the next round, the node will have a chance to become a facilitator just like any existing node.
To leave the network, nodes simply stop submitting CP blocks.
There is a subtlety here which happens when the node is elected as a facilitator in the following round.
In this case, the node must fulfil its obligation by completing the consensus protocol, but without proposing its own CP block, before leaving.
Otherwise, the $n \ge 3t + 1$ condition may be violated.

\subsection{Optimising Validation Protocol Using Cached Agreed Fragments}
\label{sec:caching}
One more way to improve the efficiency of $\Pi_v$ is to use a single agreed fragment to validate multiple transactions.
Concretely, for node $A$, upon receiving an agreed fragment from node $B$,
rather than validating a single transaction,
$A$ attempts to validate all transactions performed with $B$, which are in the unknown state but also in that fragment.

The benefit of this technique is maximised when a node only transacts with one other node.
In this case, the communication of one fragment is sufficient to validate all transactions in that fragment.
In the opposite extreme, if every transaction that the node makes is with another unique node,
then the caching mechanism would have no effect.

\subsection{Total Fork Detection}
\label{sec:fork}
The validation algorithm guarantees that there are no forks within a single agreed fragment,
which is sufficient for some applications such as proving the existence of some information.
However, for applications such as cryptocurrency where every block depends on one or more previous blocks,
our scheme is not suitable.
For such applications, we need to guarantee that there are no forks from the genesis block leading up to the TX block of interest.

We offer two approaches to do total fork detection.
First and the easiest solution is to ask for the complete hash chain of the counterparty.
The verifier can be sure that there are no forks if the following conditions hold.
\begin{enumerate}
    \item The hash pointers are correct.
    \item All the CP blocks are in consensus.
    \item The TX of interest is in the chain.
\end{enumerate}
We use this approach in our prior work on Implicit Consensus~\cite{implicitconsensus}.
Nodes employ caching to minimise communication costs, and we call this effect spontaneous sharding.

The second approach is probabilistic but with only a constant communication overhead over our current design.
For a node, observe that if all of its agreed fragments has a transaction with an honest node,
then the complete chain is effectively validated in a distributed manner.
The only way for an attacker to make a fork is to ensure that the agreed fragment containing the fork has no transactions with honest nodes.
Such malicious behaviour is prevented probabilistically using a challenge-response protocol as follows.
Suppose node $A$ wish to make a transaction with node $B$.
$A$ first sends a challenge to $B$ asking it to prove that it holds a valid agreed fragment between some consensus round specified by $A$.
If $B$ provides a correct and timely response, then they run the transaction protocol as usual.
Otherwise, $A$ would refuse to make the transaction.

