\section{Conclusion}

In this work we introduced an application neutral blockchain system which we call Checo.
We add checkpoint block to the existing TrustChain data structure for use in our consensus protocol.
The round based consensus protocol uses ACS as a building block to reach consensus on checkpoint blocks.
The consensus result, which is a set of checkpoint blocks, lets nodes elect new facilitators and create new checkpoint blocks.
To make transactions, nodes use the transaction protocol,
which is adapted from an existing work called True Halves.
Finally, we introduce a validation protocol which ensures that if agreed fragments for some transaction exists,
then nodes reach agreement on the validity of that transaction.
The novelty of the validation protocol is that it uses point-to-point communication,
i.e. nodes only validate the transactions of interest,
this enables our horizontal scalability property.

The research question we asked in~\Cref{ch:problem} is the following.
\begin{displayquote}
\emph{Is it possible to design a blockchain consensus protocol that is fault tolerant, scalable and can reach global consensus?}
\end{displayquote}
We answer it in the affirmative.
Fault tolerance is guaranteed if $n \ge 3t + 1$ by using ACS as a building block.
While $t$ may be small compared to the population size $N$,
we show that the probability for the system to fail is low even when $n \ge 3t + 1$ does not hold as long as the proportion of malicious nodes is not close to a third of $N$.
For example, if there the population size is 1000 and 20\% of the nodes are malicious,
the probability for a round to potentially to fail is bounded below $2.6 \times 10^{-16}$.
This probability bound would decrease as the population size increases.
Horizontal scalability property is demonstrated both analytically and experimentally.
Unlike sharding protocols, the property holds regardless of transaction characteristics and needs no parameter selection.
Finally, we achieve global consensus on transactions via consensus on checkpoint blocks.

This work is the first step in building a new paradigm for blockchain consensus protocol.
It has the potential to efficiently cultivate trust on the internet in the presence of faults without a central authority.
We hope to improve our design by building a concrete application on top of it.