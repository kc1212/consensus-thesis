\section{Problem Description}
\label{sec:description}
We introduce the problem as a modified Byzantine consensus problem.
In our model, we consider $N$ nodes, $t$ of which are Byzantine.
Nodes in our system make transactions with each other.
Transactions can be in one of three states---\emph{valid}, \emph{invalid} and \emph{unknown}, defined by some function $\textsf{get\_validity}(\cdot)$.
We seek a protocol that defines these states given a transaction,
but also satisfies the following properties.
\begin{itemize}
    \item \emph{Agreement}:
        If any correct node decides on the validity (except when it is \emph{unknown}) of a transaction,
        then all other correct nodes are able to reach the same conclusion or \emph{unknown}.
    \item \emph{Validity}:
        The protocol outputs the correct result according to $\textsf{get\_validity}(\cdot).$
    \item \emph{Scalability}:
        If every node makes transactions at the same rate,
        then as $N$ increases,
        the global transaction rate should increase linearly w.r.t. $N$.
\end{itemize}

Note that the agreement and validity property are similar to what is often seen in Byzantine consensus problems but there are subtle differences and relaxations.
Firstly, the agreement property only holds if honest nodes do not output \emph{unknown}.
For example, it is ok if two honest nodes output \emph{valid} and \emph{unknown}, but it is not ok to output \emph{valid} and \emph{invalid}.
Secondly, the validity is a relaxation of what is typically seen in Byzantine consensus problems,
namely we do not make any guarantee on whether the transaction is \emph{valid} even if all nodes are honest,
only that it is correct according to the $\textsf{get\_validity}(\cdot)$ defined by the protocol.

Our problem is purposefully made to be application neutural,
i.e. there are no constraints on the semantics of transaction.
This is so that the protocol can act as a backbone to many applications.
Due to this fact, we also do not consider global fork prevention or detection,
as some application may not need such strong guarantees such as recording the number of bytes uploaded and downloaded in Tribler~\cite{pimotte, pouwelse2008tribler}.
On the other hand, we give an alternative construction that does have fork detection in~\Cref{sec:fork}.

\textbf{System model.}
We assume purely asynchronous channels with eventual delivery.
Thus in no stage of the protocol are we allowed to make timing assumptions.
The adversary has full control of the delivery schedule and the message ordering of all messages.
But they are not allowed to drop messages except for their own.
Reliability can be achieved in unreliable networks by resending messages or using some error correction code.

\textbf{Security assumptions.}
The malicious nodes are Byzantine and we use a static, round-adaptive corruption model.
That is, if a round has started, the corrupted nodes cannot change until the next round.
We assume there exists a Public Key Infrastructure (PKI), and nodes are identified by their unique and permanent public key.
Finally, we use the random oracle (RO) model, i.e. calls to the random oracle are denoted by $\textsf{H}: \{0, 1\}^* \rightarrow \{0, 1\}^\lambda$,
where $\{0, 1\}^*$ denotes the space of finite binary strings and $\lambda$ is the security parameter.
Under the RO model, the probability of successfully computing the inverse of the hash function is negligible with respect to $\lambda$~\cite{bellare1993random}.

