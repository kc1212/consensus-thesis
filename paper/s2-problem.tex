\section{Problem Description}
\label{sec:description}
We introduce the problem as a modified Byzantine consensus problem.
The modification is primarily derived from the need of horizontal scalability,
which is not a part of a typical Byzantine consensus problem.
In our model, we consider $N$ nodes, $t$ of which are Byzantine.
Nodes in our system make transactions with each other.
Transactions can be in one of three states---\emph{valid}, \emph{invalid} and \emph{unknown}.
We seek a protocol that satisfies the following properties.
\begin{itemize}
    \item \emph{Agreement}:
        If any correct node decides on the validity of a transaction, except when it is \emph{unknown},
        then all other correct nodes are able to reach the same conclusion or decide \emph{unknown}.
    \item \emph{Validity}:
        If a transaction is valid, then it must have been created by two honest nodes.
    \item \emph{Scalability}:
        If every node makes transactions at the same rate,
        then as $N$ increases,
        the global transaction rate should increase linearly w.r.t. $N$.
\end{itemize}

Note that the agreement property is similar, but a relaxed version of what is often seen in a Byzantine consensus problem.
Namely, the property only holds if honest nodes do not output \emph{unknown}.
For example, for a transaction, it is fine if two honest nodes output \emph{valid} and \emph{unknown}, but they should never output \emph{valid} and \emph{invalid}.
Our problem does not have a termination property.
Instead, nodes are incentivised to complete the protocol execution otherwise they risk economical loss;
we describe this phenomenon in~\Cref{sec:correctness-of-validity}.

The problem is purposefully made to be application neutral,
i.e. there are no constraints on the semantics of transactions.
This formulation is so that the protocol can act as a building block to many applications.
Thus, we do not consider global fork prevention or detection,
as some application may not need such strong guarantees such as the accounting of internet traffic in Tribler~\cite{pimotte, pouwelse2008tribler}.
On the other hand, we give two alternative constructions that do perform fork detection in~\Cref{sec:fork}.
% TODO our transactions are rather different from a typical cryptocurrency transaction,
% should we still call it transactions or something else?

\textbf{System model.\quad}
We assume purely asynchronous channels with eventual delivery.
Thus in no stage of the protocol are we allowed to make timing assumptions.
The adversary has full control of the delivery schedule and the message ordering of all messages.

\textbf{Security assumptions.\quad}
The malicious nodes are Byzantine,
meaning that there are no restrictions on the type of failure.
We use a static, round-adaptive corruption model.
That is, if a round has started, the corrupted nodes cannot change until the next round.
We assume there exists a Public Key Infrastructure (PKI), and nodes are identified by their unique and permanent public key.
Finally, we use the random oracle (RO) model, i.e. calls to the random oracle are denoted by $\textsf{H}: \{0, 1\}^* \rightarrow \{0, 1\}^\lambda$,
where $\{0, 1\}^*$ denotes the space of finite binary strings and $\lambda$ is the security parameter.
Under the RO model, the probability of successfully computing the inverse of the hash function is negligible with respect to $\lambda$~\cite{bellare1993random}.

