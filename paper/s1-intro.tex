\section{Introduction}

% problem description
The first blockchain system---Bitcoin---is almost ten years old. 
Its market capitalisation is over \$60 billion USD and growing~\cite{bitcoinmarketcap}.
We can be reasonably sure that such systems,
even if their application is still somewhat limited,
are here to stay in the foreseeable future.
Driven by the success of Bitcoin, we see a renaissance of consensus research~\cite{luu2016elastico, kogias2016enhancing, kokoris2017omniledger},
where the primary focus is to improve the scalability of blockchain systems,
which is due to the inefficiencies of the consensus mechanism---proof-of-work (PoW).
For example, Bitcoin can only do 7 transactions per second (TPS) at most~\cite{vukolic2015quest}.
While adjusting the block size (which Bitcoin has recently done via SegWit~\cite{segwit}) and/or the block interval may increase TPS,
it also leads to centralisation as larger blocks take longer to propagate through the network,
putting miners that do not have a fast network at a disadvantage~\cite{croman2016scaling}.
Furthermore, due to the bandwidth and latency of today's network,
it is not possible to achieve more than 27 TPS from simply adjusting the block size or block interval~\cite{croman2016scaling}.

\textbf{Related work.\quad}
Many approaches exist for improving the scalability of early blockchain systems.
Off-chain transactions make use of the fact that if nodes make frequent transactions,
then it is not necessary to store every transaction on the blockchain,
only the net settlement is needed.
The best examples are Lightning Network~\cite{lightningnetwork} and Duplex Micropayment Channels~\cite{decker2015fast}.
It promises significant scalability improvements, but complicates user experience and leads to centralisation.
Each node must deposit a suitable amount of Bitcoins into a multi-signature account.
A low deposit would not allow large transactions.
A high deposit locks the user from using much of her Bitcoins outside the channel.
In addition, the user must proactively check whether the counterparty has broadcasted an old channel state so that the user does not lose Bitcoins.
Moreover, creating channels with sufficient balance and also keeping it online to act as a router is expensive.
A casual user is not capable of such tasks, leading to centralisation.
% TODO other side-chain schemes? polkadot? perum? dfinity?

Another way to improve transaction rate is to use traditional Byzantine consensus algorithms such as PBFT~\cite{castro1999practical} in a permissioned ledger such as Hyperledger Fabric~\cite{cachin2016architecture}.
In essence, such systems contain a fixed set of nodes (sometimes called validating peer) that run a Byzantine consensus algorithm to decide on new blocks.
Such systems can achieve much higher transaction rates, e.g., 10,000 TPS if the number of validating peer is under 16 for PBFT~\cite[Section 5.2]{miller2016honey}.
However, these systems do not scale, e.g., the transaction rate drops to under 5000 TPS when the number of validating peer is 64~\cite[Section 5.2]{miller2016honey}.
Moreover, the validating peers are predetermined which makes the system unsuitable for the open internet.

Recent research has developed a class of hybrid systems which uses PoW for committee election,
and Byzantine consensus algorithms to agree on transactions, e.g., ByzCoin~\cite{kogias2016enhancing} and Solidus~\cite{abraham2016solidus}.
This design is primarily for permissionless systems because the PoW leader election aspect prevents the Sybil attack~\cite{douceur2002sybil}.
It overcomes the early blockchain scalability issue by delegating the transaction validation to a Byzantine consensus protocol.
A tradeoff of such systems is that they cannot guarantee a high level of fault tolerance when there is a large number of malicious nodes (but less than a majority).
ByzCoin and Solidus all have some probability of electing more than $t$ Byzantine nodes into the committee,
where $t$ is typically just under a third of the committee size (a lower bound of Byzantine consensus~\cite{pease1980reaching}).
Again, because these systems must reach consensus on all transactions,
none of them achieves horizontal scalability.

Finally, a technique that does achieve horizontal scalability is sharding, e.g., Elastico~\cite{luu2016elastico} and OmniLedger~\cite{kokoris2017omniledger}.
It involves grouping nodes into multiple committees of constant size, also known as shards,
and nodes within a single shard run a Byzantine consensus algorithm to agree on a set of transactions that belong to that specific shard.
The number of shards grows linearly with respect to the total computational power of the network;
hence the transaction rate also grows linearly.
The limitation of sharding is that it is only optimal if transactions stay in the same shard.
In fact, Elastico cannot atomically process inter-shard transactions.
OmniLedger has an inter-shard transaction protocol but choosing a shard size that matches the transaction characteristics of the network is difficult.
An inadequate shard size would result in a large number of inter-shard transactions which would hinder scalability.
% The authors of OmniLedger noted that inter-shard transactions have significantly higher latency compared to the consensus protocol~\cite{kokoris2017omniledger}.

\textbf{Research question.\quad}
Thus far, there are no systems that achieve horizontal scalability in the general case,
which leads to the goal of this work.
Hence the research question which we wish to answer is as follows.
\begin{displayquote}
    \emph{How can we design a horizontally scalable blockchain consensus protocol?}
    \end{displayquote}
Concretely, a blockchain consensus protocol should be application neutral.
For example, PoW is application neutral because transaction semantics does not affect it,
i.e. it can be applied in different applications such as cryptocurrency (Bitcoin) and domain name system (Namecoin~\cite{namecoin}).
Further, we are interested in horizontal scalability in the general case as it enables ubiquitous use.
That is, adding more nodes to the network should result in higher transaction throughput.

\textbf{Contribution.\quad}
The key insight is not to reach consensus using an existing consensus algorithm (a modified HoneyBadgerBFT~\cite{miller2016honey}) on transactions themselves,
but on special blocks called checkpoint blocks,
such that transactions are nevertheless verifiable at a later stage by any node in the network.
Our main contributions are the following.
\begin{itemize}
    \item We formally introduce a blockchain system---\textsc{Checo}\footnote{Derived from ``\emph{Che}ckpoint \emph{co}nsensus''.}.
        It uses individual hash chains and checkpoints on every node to achieve
        horizontal scalability in the general case for the first time.
    \item We analyse \textsc{Checo} to ensure correctness according to our definition.
    \item We provide an implementation and then experiment with up to 1200 nodes,
        our results show strong evidence of horizontal scalability.
\end{itemize}

\textbf{Roadmap.\quad}
In~\Cref{sec:description},
we give the problem description and our system model.
\Cref{sec:system} gives the formal system architecture.
In~\Cref{sec:tradeoffs}, we discuss a few design variations and their tradeoffs.
We argue the correctness and fault tolerance properties of our system in~\Cref{sec:analysis}.
Then we evaluate our system experimentally in~\Cref{sec:implementation}.
Finally, we conclude our work in~\Cref{sec:conclusion}.
