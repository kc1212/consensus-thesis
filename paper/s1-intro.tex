
\section{Introduction}

% problem description
Bitcoin is almost 10 years old and its market capitalisation is over \$60 billion USD and growing~\cite{bitcoinmarketcap}.
We can be reasonably sure cryptocurrencies,
even if their application is limited,
are here to stay in the foreseeable future.
Driven by the success of Bitcoin, we are seeing a renaissance of consensus research~\cite{miller2016honey, kogias2016enhancing, kokoris2017omniledger},
where the primary focus is to improve the scalability of blockchain system.
This is due to the consensus mechanism in early blockchain systems---proof-of-work (PoW).
For example, Bitcoin can only do 7 transactions per second (TPS) at most~\cite{vukolic2015quest} prior to SegWit,
which introduces a new block structure.
While adjusting the block size and/or the block interval may increase TPS,
it also leads to centralisation as larger blocks take longer to propagate in the network,
giving miners that do not have a fast network a disadvantage~\cite{croman2016scaling}.
In today's network, it is not possible to achieve more than 758 TPS
if new blocks need to be propogated timely to 90\% of the network.

% research question
In this work, answer the following research question.
\begin{displayquote}
\emph{Is it possible to design a blockchain consensus protocol that is fault-tolerant and horizontally scalable?}
\end{displayquote}
A blockchain consensus protocol should be application neutural,
for example PoW without the transaction logic.
Our system should be Byzantine fault tolerant up to some threshold.
The threshold may be made adjustable by trading off performance.
Finally, we are interested in horizontal scalability as it enables ubiquitous use.
That is, by adding new nodes into the network, the global transaction rate should increase proportionally.

% contribution
The key insight is to \emph{not} reach consensus using an existing consensus algorithm (a modified HoneyBadgerBFT~\cite{miller2016honey}) on transactions themselves but on special blocks called checkpoint blocks,
such that transactions are nevertheless verifiable at a later stage by any node in the network.
Our main contributions are the following.
\begin{itemize}
    \item We formally introduce a blockchain system---Checo\footnote{\emph{Che}ckpoint \emph{co}nsensus}.
        It uses individual hash chains and checkpoints on every node to achieve
        horizontally scalable for the first time.
    \item We analyse Checo to ensure correctness as defined in our architecture.
    \item We provide an implementation and then experiment with up to 1200 nodes,
        our results show strong evidence of horizontal scalability.
\end{itemize}

% organisation
In~\Cref{sec:related},
we cover the current state-of-the-art and the necessary background.
\Cref{sec:system} gives the formal system architecture.
Next, we argue the correctness and fault tolerance properties of our system in~\Cref{sec:analysis}.
We evaluate our system experimentally in~\Cref{sec:implementation}.
Finally, we conclude our work in~\Cref{sec:conclusion}.