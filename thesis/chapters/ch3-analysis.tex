\chapter{Analysis}
\label{ch:analysis}

Up to this point we described our system specification in detail.
Of course, specification along does not establish any truths.
In this chapter, we prove two aspects of our system.
First is correctness,
where we show that the consensus protocol and the validation protocol satisfies their desired properties
(\Cref{def:consensus} and \Cref{def:validation} respectively).
The second is performance,
where we prove the lower bound of our throughput and show that it out performs classical blockchain systems.

\section{Correctness}
Our first objective in this section is to establish truths regarding the correctness of our protocol.
We do this in two parts.
First we use mathematical induction to show that properties in \Cref{def:consensus} holds for all round.
Building on top that, if \Cref{def:consensus} is true, then we can show that many properties in \Cref{def:validation} is true.

\textbf{\emph{Correctness of consensus}}

$\forall r \in \mathbb{N}$, the following properties must be satisfied.
\begin{enumerate}
    \item \emph{Agreement}:
        If one honest node outputs a list of facilitators $F_r$,
        every other node honest decides on $F_r$
    \item \emph{Validity}:
        If any honest node outputs $F_r$, then $|F_r| > n$ and $|F_r|$ must contain at least $n - t$ honest nodes.
    \item \emph{Fairness}:
        Every node should have an equal probability of becoming a facilitator.
    \item \emph{Termination}:
        At the end of the round, every node outputs some $F_r$.
\end{enumerate}


\subsection{Correctness of Consensus}

We begin our analysis by establishing the fact that the $\textsf{get\_facilitator}(\cdot)$ is fair.
\begin{lemma}
\label{lemma:fairness}
\textbf{\emph{Fairness of facilitator election}}

Every node with a CP block in $\C_r$, should have an equal probability to be elected as a facilitator.
\end{lemma}
\begin{proof}
This directly follows from the random oracle model.
Recall that the luck value is computed using $\textsf{H}(\C_r, || pk_i)$.
Since $pk_i$ is unique for every node that has a CP block in $\C_r$, the output of $\textsf{H}(\cdot)$ is uniformly random.
This implies that the ordered sequence by luck value is uniformly random.
\end{proof}


Using \Cref{lemma:fairness}, we show that our consensus protocol satisfied \Cref{def:consensus}.
\begin{lemma}
\textbf{\emph{Correctness of consensus}}
$\forall r \in \mathbb{N}$, agreement, validity, fairness and termination holds.
\end{lemma}

\begin{proof}
We proof by mathematical induction.

In the base case, agreement, validity fairness and termination follows directly from the bootstrap protocol, due to the bootstrap oracle.
Note that the result is $F_1$, which indicates the facilitators that are agreed in round 1 and are responsible for driving the ACS protocol in round 2.

For the inductive step, we assume that the properties hold for round $r$.
Then, to start round $r + 1$, the honest nodes begin sending CP blocks to $F_r$. % TODO but they don't send at the same time!
Since the honest region in $F_r$ waits for $D$ and $D \gg \Delta$,
the CP blocks of the honest nodes are guaranteed to be received by at least $n - t$ facilitators.
The agreement property of ACS (from \Cref{def:acs}) ensures that the consensus result $\C_{r + 1}$ is in consensus. 
The validity property of ACS ensures that $\C_{r + 1}$ contains the input of at least $n - 2t$ parties, but this is a quorum containing at least one honest facilitator,
thus $\C_{r + 1}$ contains the CP blocks of all facilitators.
Observe that $F_{r + 1}$ is computed using the deterministic function $\textsf{get\_facilitators}(\cdot)$.
Thus agreement of $F_{r + 1}$ follows directly from agreement of ACS.
Due to the assumption that the adversary cannot corrupt more than $t$ nodes,
validity of $F_{r + 1}$ also follows from validity of ACS.
The fairness property follows from \Cref{lemma:fairness}.
Finally, the termination property holds because $D$ eventually elapses and then ACS eventually terminates.
This completes our proof.


\end{proof}


\subsection{Correctness of Validation}

Remark: chain structure cannot be satisfied, but can be probabilistically checked

\section{Performance}