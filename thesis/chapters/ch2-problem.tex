\chapter{Problem description}
\label{ch:problem}

From the Introduction, we saw that early blockchain systems face a scalability problem.
In particular, the use proof-of-work is unlikely to reach a transaction rate suitable for ubiquitous use.
Many attempts to fix the scalability problem exist in academia as well as the industry.
In this chapter, we describe the related work in the form of a mini taxonomy and identify areas for improvement and limitations.
Then we state our research question which aims to overcome some of the limitations.

\section{Related work and mini taxonomy}
\label{sec:taxonomy}

Blockchain research has seen a surge in recent years from both the industry and academia,
this resulted in a myriad of blockchain systems with different approaches to scalability.
To understand the current state-of-the-art,
we classify various blockchain systems by their scalability approach.

\subsection{Early blockchain systems}
This category represents the baseline,
which are systems with a probabilistic consensus algorithm.
That is, transactions never reach consensus with a probability of 1.
Given sufficient computational power, adversaries are able to undo any transaction.
The typical examples are proof-of-work based systems such as Bitcoin, Ethereum and other Altcoins.
In Bitcoin, nodes compete in solving puzzles.
The first node solves the puzzle has the privilege to create a new block.
The level of consensus of a block containing many transactions is determined by how deep it is in the Bitcoin blockchain,
also called the number of confirmation.
The probability of a block being orphaned drops exponentially as the depth increases~\cite{bitcoin}.
Nevertheless, the probability of the highest block being orphaned is non-negligible.

The advantage of this approach is that it can be used in a large network.
In fact, prior to Bitcoin, there was no internet-scale consensus protocol.
Bitcoin is also fault-tolerant because attackers can not out pace honest users in finding new blocks unless they have a majority of the hash power.
As long as the majority is honest and reasonable measures are taken to prevent the selfish mining attack~\cite{eyal2014majority}.
The disadvantage, however, is that transactions are never in consensus with a probability of 1---no consensus finality.
Hence double spending is possible for transactions that do not have a large number of confirmations.
Furthermore, as we mentioned in the~\Cref{ch:intro},
the performance is limited due to the fact that blocks are of fixed size and are generated at fixed intervals.
Thus early blockchain systems do not achieve horizontal scalability.

% Our system significantly improves upon Bitcoin and other classical blockchain systems in performance, scalability and consensus finality.
% The results described in \Cref{ch:analysis} and \Cref{ch:implementation}, show that we have horizontal scalability,
% where more nodes result in more global throughput.
% Further, we do not have the aforementioned probabilistic behaviour,
% once some consensus result is decided, it cannot be orphaned, thus our consensus is final.
% The leadership election is also not ideal in classical blockchain systems.
% Mining can be seen as a technique to elect a single leader.
% The leader has full control of what goes into the block thus it may selectively censor transactions.
% We use ACS, so as long as the CP block is in $n - 2t$ nodes, it is guaranteed to be in consensus.

% However, the security aspect falls short of the ``honest majority'' security model that classical blockchains claim to 
% have\footnote{Recently it was shown that doing selfish-ming would give the adversary an unfair advantage when she only controls 25\% of the mining power~\cite{eyal2014majority}}.
% Our system risks going into erroneous state if the inequality $n \ge 3t + 1$ is not satisfied.
% Classical blockchain systems also have an incentive mechanism, thus they do not depend on altruistic nodes and encourages participation.
% Our system on the other hand does not have an incentive mechanism because we make no assumptions on the application.

\subsection{Off-chain transactions and payment networks}
Off-chain transactions make use of the following fact.
If nodes make frequent transactions,
then it is not necessary to store every transaction on the blockchain,
only the net settlement is necessary.
The best examples are Lightning Network~\cite{lightningnetwork} and Duplex Micropayment Channels~\cite{decker2015fast}.

Off-chain transaction systems are implemented using multi-signature addresses~\cite{bitcoinmultisig} and hashed time-locked Bitcoin contracts~\cite{bitcointimelock}.
In the simplest case, 
if two parties wish to make transactions,
they open a payment channel in the form of a multi-signature Bitcoin address (for two parties it would be a 2-of-2 signature address).
Each of the party must also deposit some Bitcoins in the multi-signature address,
this is called the opening transaction.
Both parties keep the channel state which is not broadcasted to the network.
The state is updated when the two parties make new transactions.
The protocol disincentivises the parties from broadcasting old channel states.
If this occurs, the counterparty can drain all the Bitcoins in the channel.
To close the channel, the parties simply broadcast the latest net settlement to the Bitcoin network,
this is called the closing transaction.
Effectively, only two transactions need to be broadcasted to the Bitcoin network---the opening and the closing transaction,
even if the two parties made thousands or millions of transactions in between.
The two-party scheme can be extended to a network of channels,
which allows two parties to make off-chain transactions without an open channel as long as they are connected by nodes that do.

The advantage of such systems is that they act as add-ons to Bitcoin wish already has a large number of users.
Thus, if enough of the network wish for it (by setting a new block version),
then a large number of users will instantly benefit from it.
Some implementations already exist\footnote{For example \url{https://github.com/lightningnetwork/lnd}.},
but at the time of writing SegWit is not activated yet which is a prerequisite for Lightning Network.

On the other hand, payment channel complicates user experience and leads to centralisation.
As we mentioned earlier, each node must deposit some Bitcoins into a multi-signature account.
The users must pick a suitable amount.
If the deposit is too low it would not allow large transactions.
If the deposit is too high the user is locked out of much of its Bitcoins for use outside of the channel.
In addition, the user must proactively check whether the counterparty has broadcasted an old channel state so that the user does not lose Bitcoins.
Furthermore, creating channels with sufficient balance and also keeping it online to act as a router is expensive.
A casual user is not capable of such tasks, thus it leads to centralisation.
Payment channels, in theory, solve the scalability problem of early blockchain systems,
but to the best of our knowledge, its exact scalability characteristics are not investigated.

% Further, off-chain transactions are limited only to an exchange of cryptocurrency,
% it is less general than typical Bitcoin transactions which may be a simple smart contract.
% Time-locked contracts have a strong dependency on timing, thus disputes may arise when the payment channel is just about to close.
% Our system is purely asynchronous and make no assumption on timing,
% in fact we assume the adversary has control of the message delivery time and order.


\subsection{Permissioned systems based on Byzantine consensus}

This category of systems uses traditional Byzantine consensus algorithms such as PBFT~\cite{castro1999practical}.
In essence, they contain a fixed set of nodes (sometimes called validating peer) that run a Byzantine consensus algorithm to decide on new blocks.
This technique is often used in permissioned system where the validators must be predetermined.
For example, Hyperledger Fabric~\cite{cachin2016architecture}.

A nice aspect of Byzantine consensus and in particular PBFT is that it can handle much more transactions than classical blockchain systems.
PBFT can, for example, achieve 10,000 TX/s if the number of validating peer is under 16~\cite[Section 5.2]{miller2016honey}.
Further, in contrast to proof-of-work, PBFT consensus is final.
That is, transaction history cannot be re-written if it is already in the blockchain.

The major drawback of Byzantine consensus based systems is that it does not scale in terms of the number of validating peers.
Going back to PBFT, its transaction rate drops to under 5000 TX/s when the number of validating peer is 64~\cite[Section 5.2]{miller2016honey}.
Moreover, the validating peers are predetermined which makes the system unsuitable for the open internet.

% But our system has the potential to perform beyond that of PBFT because we represent many transactions with a single CP block,
% enabling horizontal scalability.
% Furthermore, since PBFT relies on a leader, it is not censorship resiliant.
% Our system on the other hand has the benifits of ACS where CP blocks cannot be censored.
% Finally, our system is able to work in the permissionless setting by simply submitting new CP blocks to the facilitators.
% It can even be adapted to work in the permissioned by simply removing the luck value computation.

% What we wish to have which is in Hyperledger is a smart contract system (also known as chaincodes in Hyperledger).
% We hope to design and implement smart contracts by adding additional logic to the transaction protocol and the validation protocol.
% Such functionalities we believe is better to be built into the backbone rather than having it as an add on.

\subsection{Combining proof-of-work with Byzantine consensus}
Recent research has developed a class of hybrid systems which uses PoW for committee election,
and Byzantine consensus algorithms to agree on transactions.
This design is primarily for permissionless system because the PoW leader election aspect prevents Sybils.
% And then they use a Byzantine consensus algorithm to actually reach consensus on a set of transactions within the committee.
Some examples are SCP~\cite{luu2015scp}, ByzCoin~\cite{kogias2016enhancing} and Solidus~\cite{abraham2016solidus}.

This technique overcomes the early blockchain scalability issue by delegating the transaction validation to the Byzantine consensus protocol (e.g., PBFT in ByzCoin~\cite{kogias2016enhancing}).
A major tradeoff of such systems is that they cannot guarantee correctness when there is a large number of malicious nodes (but less than a majority).
For SCP, ByzCoin and Solidus, they all have some probability to elect more than $t$ Byzantine nodes into the committee,
where $t$ is typically just under a third of the committee size (a lower bound of Byzantine consensus~\cite{pease1980reaching}).
This problem is especially difficult to solve because the committee is always much smaller than the population size which has more than $t$ Byzantine nodes.
Classical blockchain does not have this problem because they do not use Byzantine consensus.
Further, due to the fact that these systems must reach consensus on all transactions, none of them achieves horizontal scalability.

\subsection{Sharding}

Sharding is a technique to achieve horizontal scalability by grouping nodes into multiple committees or shards.
Nodes within a single shard run a Byzantine consensus algorithm to agree on a set of transactions that belong to that specific shard.
An intra-shard protocol is needed for transactions that involve nodes from more than one shard.
The number of shards grows linearly with respect to the total computational power in the network.
This scheme achieves horizontal scalability because if every shard commits transactions at the same throughput,
then adding more shard would naturally result in a linear increase of global throughput.
Examples of blockchain systems that use sharding are Elastico~\cite{luu2016elastico} and OmniLedger~\cite{kokoris2017omniledger}.

The biggest limitation of sharding is that it is only optimal is transactions stay in the same shard.
In fact, Elastico cannot atomically process inter-shard transactions.
OmniLedger has an inter-shard transaction protocol but choosing a good shard size is difficult.
A large shard size would make the system less scalable because the Byzantine consensus algorithm must be run by a large number of nodes.
A small shard size would result in a large number of inter-shard transactions which also hinder scalability.
The authors of OmniLedger noted that inter-shard transactions have significantly higher latency compared to the consensus protocol~\cite{kokoris2017omniledger}.
Furthermore, since no shards can be compromised for the system to function correctly,
the adversaries have more opportunities to compromise the system.

\subsection{Blockchain without global consensus}

Tangle~\cite{tangle}, Corda~\cite{corda} and the original TrustChain~\cite{multichain} do not use global consensus at all.
In this scheme, every node has their own chain.
Transactions between nodes are recorded on their respective chains.
This effectively results in a directed acyclic graph.
By avoiding global consensus, this approach achieves extremely desirable scalability properties similar to BitTorrent~\cite{cohen2003incentives}.

On the other hand, the application and security are limited.
A malicious node can easily lie to other nodes regarding its own chain by simply presenting one version of reality to a set of nodes and another version to a different set.
Thus the network will have a conflicting view on the state of the malicious node.
Applications such as digital cash can not be implemented on top of such systems because consistency is needed in solving the double spending problem.

% Our system can be considered as the same as these types of blockchains but with a lightweight consensus protocol.
% Consensus might not be applicable in all applications.
% But we believe it is important for detecting and preventing fraud.
% The example in~\Cref{fig:trustchain-bad} on page~\pageref{fig:trustchain-bad} demonstrates this.
% If $b$ makes a fork and $a$ and $c$ have no way to communicate (e.g. the adversary may control parts of the network) ,
% then $c$ is tricked to believe that her transaction with $b$ is valid.
% Only when $c$ sees a conflicting chain is she able to tell that the transaction is invalid.
% But $c$ does not know the true end-of-chain of $b$, thus she can never know whether her transaction is valid.
% This is not possible in our system because $b$ cannot convince $c$ unless he can compute exponential time algorithms
% (this is finding the second preimage for a hash function).

% However, consensus and validation comes at a cost that do not exist in Tangle, Corda or the original TrustChain.
% Our transaction rate is affected by the validation protocol in the worst case scenario as we saw in~\Cref{sec:evaluation}.

\section{Research question}

We saw in~\Cref{sec:taxonomy} that the majority of the systems cannot achieve horizontal scalability.
To the best of our knowledge, only two systems---Elastico~\cite{luu2016elastico} and OmniLedger~\cite{kokoris2017omniledger}---has experimentally demonstrated horizontal scalability.
However, these techniques rely on sharding, thus the performance highly depends on transaction characteristics and the shard size parameter because inter-shard transactions are slow.
Furthermore, all the systems described in~\Cref{sec:taxonomy} are complete blockchain systems designed for a concrete application.
We wish to design a blockchain consensus protocol that can be used in many applications.
For example, the accounting of detailed internet traffic in Tribler~\cite{pouwelse2008tribler, multichain} and decentralised market\footnote{\url{https://github.com/Tribler/tribler/blob/devel/Tribler/community/tradechain/__init__.py}}.
Hence our research question is the following.
\begin{displayquote}
\emph{How can we design a blockchain consensus protocol that is fault-tolerant,
horizontally scalable,
and able to reach global consensus?}
\end{displayquote}

In order to make sense of our research question, we must first define \emph{blockchain consensus protocol}.
Typical blockchain systems are application specific.
For example, the application in Bitcoin (any many of its derivatives) is digital cash, the application in Namecoin is domain name registration~\cite{namecoin} and the application in Siacoin is cloud storage~\cite{siacoin}.
Underneath these applications lies the blockchain consensus protocol which has the goal of reaching consensus on transactions or state changes.
In the case of Bitcoin, it is proof-of-work (PoW).
It is not concerned with the structure of the transactions,
it works by simply treating transactions as a blob of data and hashes them with the previous block.
In this work, we focus on the blockchain consensus protocol rather than any specific application.
Because it is the scalability bottleneck and is necessary for any blockchain based application.
% Also, it should be possible to build any kind of application on top our system.
% We do not impose on a consensus algorithm, as long as it satisfies the properties of atomic broadcast which we describe in~\Cref{sec:overview-cons}.

In the remainder of this section, we expand on our research question and visit each of our requirements in detail.
At the end of this section, we describe some issues that we do not consider as a part of our design,
which is the result of not focusing on any specific application.

\subsection*{Global consensus}
% WHY WE NEED GLOBAL CONSENSUS?
Having consensus is essential in blockchain systems.
It stops many types of malicious activities because the agreed state, or a representation of it, must have the consent of the honest nodes in the network.
We emphasise on the notion of a representation of the state because in contrary to traditional blockchains,
it does not necessarily mean the set of all transactions that have ever happened.
The state can be a small, but verifiable representation of all the transactions.
As an extreme example, a single digest of the complete history can represent the state.
It is even verifiable as long as there are nodes that store the history.

\subsection*{Fault-tolerance}
% WHY WE NEED FAULT-TOLERANCE
Similar to many blockchain systems,
our system should be unaffected in the presence of powerful adversaries.
In particular, adversaries are Byzantine meaning that they can have arbitrary behaviour.
Thus anything is possible from simply omitting messages to colluding with each other in order to undermine the whole system.

There is another aspect of fault tolerance that is transaction verifiability.
All transactions must be eventually verifiable to maintain system integrity.
Further, the validation result must be consistent.
For example, if two different honest nodes attempt to verify the same transaction,
it cannot be the case that one node thinks that the transaction is valid but the other thinks it is invalid.

\subsection*{Horizontal scalability}

To enable ubiquitous use, we demand horizontal scalability.
That is, the global transaction rate should increase as more nodes join the network.
BitTorrent~\cite{cohen2003incentives} is an example of such a system,
where peers interact with each other to exchange files without any global bottleneck.
% Early blockchain systems may run in a network of thousands of nodes, but the performance does not scale.
% On the other hand, as we will see in~\Cref{sec:acs-background}, Byzantine consensus algorithms perform well but only when in a small network.
% In this work, our goal is to overcome the limitations of early blockchain systems and Byzantine consensus algorithms to create a horizontally scalable blockchain.

\subsection*{Limitations}
Since our consensus protocol is application neutral, it does not attempt to prevent the Sybil attack~\cite{douceur2002sybil}.
A good defence mechanism requires application specific information.
For example, a social network based Sybil defence mechanisms use graph structure of real-world relationships~\cite{yu2006sybilguard}.
Online marketplaces such as Amazon use the rating of buyer and sellers.
% Since our system is application neutral, we do not provide an explicit Sybil defence mechanism.
On the other hand, our system also has no restrictions on the Sybil defence mechanism
and applications built on top of our system can use the best mechanism for their purpose.

For the same reason, we do not explicitly consider spam or denial of service attacks.
Without a concrete application, it is difficult to distinguish a super active user from an attacker.


% Scalability is indeed one of the key challenges we face in blockchain systems today.
% Bitcoin~\cite{bitcoin}, the largest permissionless\footnote{Explain permissionless} blockchain system
% in terms of market capitalisation~\cite{bitcoinmarketcap} has a maximum transaction rate of merely 7 transaction per second (TX/s).
% This is due to the consensus mechanism in Bitcoin, namely proof-of-work (PoW),
% miners can only create new blocks every 10 minutes and every block cannot be larger than 1 megabyte.
% Payment processors in use today such as Visa can handle transaction rates in the order of thousands~\cite{visa}.
% While Bitcoin may be a revolutionary phenomenon, it clearly cannot be ubiquitous in its current state.

% An different approach is to not reach global consensus at all.
% For instance in TrustChain~\cite{multichain} and Tangle~\cite{tangle},
% nodes in the network only store their personal ledger.
% Since consensus is left out, nodes can perform transactions as fast as their machine and network allows.
% The downside of this approach is that it cannot prevent fraud (it is possible to detect fraud).
% To examplify, a malicious node Mallory may claim she has 3 units of currency to Alice,
% but in reality Mallory already spent all of it on Bob.
% If there is no global consensus and Bob and Alice never communicate,
% then the 3 units that Alice is about to receive is nonexistent.

% The scalability property of TrustChain and Tangle are exceptionally desirable.
% The global consensus mechanism of Bitcoin and many other blockchain systems are also worthwhile for detecting or preventing fraud.
% These two properties may seem mutually exclusive, but in this work, we demonstrate the opposite.
% Specifically, we answer the following research question in the affirmative.
% \emph{Is it possible to design a blockchain fabric that can reach global consensus on the state of the system 
% and also scalable?} We define scalability as a property where if more nodes join the system, then the transaction rate should increases.
% % this is not completely true, there's a cap
