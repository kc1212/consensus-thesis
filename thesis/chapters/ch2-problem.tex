\chapter{Problem Description}
\label{ch:problem}

- long history of BFT, since Lamport
- PBFT
- PBFT triggered a renaissance in BFT replication research, with protocols like Q/U

- classical blockchains are nice, but is it possible to also use 30 years of BFT research?
- non-consensus chains scales, but no consensus...
- alternative consensus model

% Scalability is indeed one of the key challenges we face in blockchain systems today.
% Bitcoin~\cite{bitcoin}, the largest permissionless\footnote{Explain permissionless} blockchain system
% in terms of market capitalisation~\cite{bitcoinmarketcap} has a maximum transaction rate of merely 7 transaction per second (TX/s).
% This is due to the consensus mechanism in Bitcoin, namely proof-of-work (PoW),
% miners can only create new blocks every 10 minutes and every block cannot be larger than 1 megabyte.
% Payment processors in use today such as Visa can handle transaction rates in the order of thousands~\cite{visa}.
% While Bitcoin may be a revolutionary phenomenon, it clearly cannot be ubiquitous in its current state.

% An different approach is to not reach global consensus at all.
% For instance in TrustChain~\cite{multichain} and Tangle~\cite{tangle},
% nodes in the network only store their personal ledger.
% Since consensus is left out, nodes can perform transactions as fast as their machine and network allows.
% The downside of this approach is that it cannot prevent fraud (it is possible to detect fraud).
% To examplify, a malicious node Mallory may claim she has 3 units of currency to Alice,
% but in reality Mallory already spent all of it on Bob.
% If there is no global consensus and Bob and Alice never communicate,
% then the 3 units that Alice is about to receive is nonexistent.

% The scalability property of TrustChain and Tangle are exceptionally desirable.
% The global consensus mechanism of Bitcoin and many other blockchain systems are also worthwhile for detecting or preventing fraud.
% These two properties may seem mutually exclusive, but in this work, we demonstrate the opposite.
% Specifically, we answer the following research question in the affirmative.
% \emph{Is it possible to design a blockchain fabric that can reach global consensus on the state of the system 
% and also scalable?} We define scalability as a property where if more nodes join the system, then the transaction rate should increases.
% % this is not completely true, there's a cap

% Using the aforementioned insight,
% we explore an alternative consensus model for blockchain systems where transactions themself do not reach consensus,
% but nevertheless verifiable at a later stage by any node in the network.
% Informally, out model works as follows.
% Every node stores its own blockchain and every block is one transaction, same as the TrustChain construction.
% We randomly selected nodes in every round, the selected ones are called facilitators.
% They reach consensus not on the individual transactions,
% but on the state of every chain represented by a single digest, we call this state the checkpoint.
% If a checkpoint of some node is in consensus, 
% then that node can proof to any other node that it holds a set of transactions that computes (form a chain) to the checkpoint.
% This immediately show that those transactions are tamper-proof.
