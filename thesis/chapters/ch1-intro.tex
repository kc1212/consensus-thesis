TODO some positive intro about blockchain

One of the key issues in many blockchain systems today is that they are not scalable.
Bitcoin~\cite{bitcoin}, the largest permissionless\footnote{Explain permissionless} blockchain system
in terms of market capitalisation~\cite{bitcoinmarketcap} has a maximum transaction rate of merely 7 transaction per second (TX/s).
This is due to the consensus mechanism in Bitcoin, namely proof-of-work (PoW),
miners can only create new blocks every 10 minutes and every block cannot be larger than 1 megabyte.
Payment processors in use today such as Visa can handle transaction rates in the order of thousands~\cite{visa}.
While Bitcoin may be a revolutionary phenomenon, it clearly cannot be ubiquitous in its current state.

An different approach is to not reach global consensus at all.
For instance in TrustChain~\cite{multichain} and Tangle~\cite{tangle},
nodes in the network only store their personal ledger.
Since consensus is left out, nodes can perform transactions as fast as their machine and network allows.
The downside of this approach is that it cannot prevent fraud (it is possible to detect fraud).
To examplify, a malicious node Mallory may claim she has 3 units of currency to Alice,
but in reality Mallory already spent all of it on Bob.
If there is no global consensus and Bob and Alice never communicate,
then the 3 units that Alice is about to receive is nonexistent.

The scalability property of TrustChain and Tangle are exceptionally desirable.
The global consensus mechanism of Bitcoin and many other blockchain systems are also worthwhile for detecting or preventing fraud.
These two properties may seem mutually exclusive, but in this work, we demonstrate the opposite.
Specifically, we answer the following research question in the affirmative.
\emph{Is it possible to design a blockchain fabric that can reach global consensus on the state of the system 
and also scalable?} We define scalability as a property where if more nodes join the system, then the transaction rate should increases.
% this is not completely true, there's a cap

Our primary insight came from observing the differences between how transactional systems work in the real world and how they work in blockchain systems like Bitcoin. 
Take a restaurant owner for example, most of the time the customer is honest and pays the bill.
There is no need for the customer or the restaurant owner to report the transaction to any central authority 
because both parties are happy with the transaction.
On the other hand, if the customer leaves without paying the bill,
then the restaurant owner would report the incident to some central authority, e.g. the police.
On the contrary, in blockchain systems, every transaction is effectively sent to the miners,
which can be seen as a collective authority.
This consequently lead to limited scalability because every transaction must be validated by the authority even when most of the transaction are legitimate.

Using the aforementioned insight,
we explore an alternative consensus model for blockchain systems where transactions themself do not reach consensus,
but nevertheless verifiable at a later stage by any node in the network.
Informally, out model works as follows.
Every node stores its own blockchain and every block is one transaction, same as the TrustChain construction.
We randomly selected nodes in every round, the selected ones are called facilitators.
They reach consensus not on the individual transactions,
but on the state of every chain represented by a single digest, we call this state the checkpoint.
If a checkpoint of some node is in consensus, 
then that node can proof to any other node that it holds a set of transactions that computes (form a chain) to the checkpoint.
This immediately show that those transactions are tamper-proof.

We begin the detailed discussion by formulating the model \Cref{sec:model}. 
Next, we analyse the correctness, security and performance of our design in \Cref{sec:analysis},
this is where we present our theorems.
Implementation and experimental results are discussed in \Cref{sec:implementation}.
Finally, we compare our system with other state-of-the-art blockchain systems and conclude in Sections \ref{sec:related} and \ref{sec:conclusion} respectively.


%%% Local Variables:
%%% mode: latex
%%% TeX-master: "../thesis"
%%% End: