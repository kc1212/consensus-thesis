\chapter{Conclusion}
\label{ch:conclusion}

The research question we asked in~\Cref{ch:problem} is the following.
\begin{displayquote}
\emph{Is it possible to design a blockchain consensus protocol that is fault tolerant, scalable and can reach global consensus?}
\end{displayquote}
We answer it in the affirmative and our system achieves all the requirements.
Fault tolerance is guaranteed if $n \ge 3t + 1$ from ACS.
While $t$ may be small compared to the population size $N$,
we show that the probability for the system to fail is low even when $n \ge 3t + 1$ does not hold as long as the proportion of malicious nodes is not close to a third of $N$.
Horizontal scalability is demonstrated both analytically and experimentally.
With 1200 nodes in the network we were able to achieve nearly 5000 TX/s---an order of magnitude improvement over Bitcoin.
Finally, we achieve global consensus on transactions via consensus on checkpoint blocks.

This work is the first step in building a new paradigm for blockchain consensus protocol.
It has the potential to efficiently cultivate trust on the internet in the presence of faults without any central authority.
We hope to improve our design by building a concrete application on top of it.

\section{Future work}
While our system has excellent scalability properties, it is not without limitations.
Much of it is from the fact that it does not have a concrete application.
We do not attempt to prevent the Sybil attack, spam or DoS because the accuracy of the defence depends on the understanding of the application.
For instance, without any application it would be impossible to distinguish between a very active node from a spammer because the content of the transaction ($m$) carry no meaning.

Our experiment is carried out in a somewhat idealised world (DAS-5).
To understand our system further, we hope to experiment on a real platform with real users such as Tribler~\cite{pouwelse2008tribler}.
For instance, the TrustChain in Tribler stores the number of bytes uploaded and downloaded between users, but it has no global consensus.
It would be useful to integrate our consensus protocol into Tribler to prevent misreporting attacks.
More importantly it gives us an opportunity to observe the characteristics of our system in the real world and how it evolves with new features or alterations.

The fault tolerance property of our system is lower than that of typical blockchain systems,
which usually aim for fewer than 50\%.
For future work, we propose the following two ways to improve fault tolerance.
First, use a reputation mechanism to select facilitators rather than simply selecting them randomly so that faulty nodes are less likely to be selected.
This technique of course heavily depends on the application,
for P2P file sharing systems the reputation score may be the number of bytes contributed to the network.
Second, instead of using ACS we use a proof-of-stake (PoS) scheme which means that nodes with more stake in the system gets more vote.
In an ideal PoS scheme, the system can tolerate any fault if the majority of the stake holders are honest.

Finally, as mentioned in~\Cref{sec:permissionless}, we wish to evaluate our system (analytically and experimentally) in the permissionless settings.
This would enable our protocol to be applied to a much wider range of applications.
