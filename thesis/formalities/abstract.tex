%De aankondiging bevat de spreker, titel, plaats, datum en tijd, samenstelling van de afstudeercommissie en een korte samenvatting (maximaal 25 regels).
\thispagestyle{empty}

\noindent \textbf{Author}\\
\begin{tabular}{l}
Kelong Cong\\
\\
\end{tabular}\\
\noindent \textbf{Title}\\
\begin{tabular}{l}
Blockchain Consensus Protocol with Horizontal Scalability\\
\\
\end{tabular}\\
\noindent \textbf{MSc presentation}\\
\begin{tabular}{l}
% <MM> DD, YYYY (like \today)
31st August 2017\\
\\
\end{tabular}

\vspace{1.1cm}

\noindent \textbf{Graduation Committee}\\
\begin{tabular}{ll}
% The order of listing the names: Graduation prof, supervisor(s), others ordered by title + alphabetical
%examples:
Prof. dr. ir. D. H. J. Epema            & Delft University of Technology \\
Dr. ir. J. A. Pouwelse                  & Delft University of Technology \\
Dr. Z. Erkin                            & Delft University of Technology \\
\end{tabular}

\begin{abstract}
% what's the problem?
Blockchain systems have the potential to decentralise many traditionally centralised system.
However, scalability remains a key challenge.
Without a horizontally scalable solution, blockchain system remain unsuitable for ubiquitous use.
% what's the solution?
We design a novel blockchain system which we call Checo.
Each node in our system maintain individual hash chains,
which only stores transactions that involve the node.
Consensus is reached on special blocks called checkpoint blocks rather than on all the transactions.
Checkpoint blocks are effectively a hash pointer to the individual hash chains,
thus a single checkpoint block may represent a large set of transaction blocks.
The consensus protocol does not imply transaction validity.
Hence we include a validation protocol which allows nodes to verify that the counterparty correctly recorded the transaction.
% how good is the solution?
We implement a prototype and evaluate it experimentally.
Our results show strong indication of horizontal scalability even in the worst case,
where every transaction is with a randomly selected node.
For 1200 nodes, we were able to perform almost 5000 transactions per second,
orders of magnitude higher than the 7 transaction per second limit of Bitcoin.
\end{abstract}

\clearpage

